\chapter{The Enlarge Axis}\hypertarget{enlarge}{}
%\fancyhead[CE]{\scshape\color{myRed} {\addfontfeatures{Numbers=OldStyle}\thepage}\hspace{10pt}the enlarge axis}

The character recommendation of the Medieval Unicode Font Initiative (MUFI) includes a class of characters called
“Enlarged Minuscules,” for representing characters that are lowercase in shape but intermediate between lowercase 
and uppercase in size: these are often used to begin sentences in medieval manuscripts. MUFI encodes these characters 
in the Private Use Area, posing accessibility and searchability problems, as explained in the introduction to the 
“Feature Reference” chapter of this manual.

Junicode provides a solution to these problems via the OpenType feature Stylistic Set 6
(\textSourceText{ss06},\index{ss06} “Enlarged minuscules”). 
This feature also works in Junicode VF, the variable version of Junicode, which in addition offers a far more flexible 
way of representing enlarged minuscules—the Enlarge axis.

An “axis” is an aspect of a font that can be varied along a numerical range. A family of traditional fonts like Times 
New Roman has a weight axis with a font file on either end: Regular and Bold. Other font families have more weights 
along this axis: for example, Light, Medium, ExtraBold. Most variable fonts also have a weight axis, but all weights 
are contained in a single file, and users are not restricted to just a few weights, but can select any weight between 
the extremes.

Because almost every font family has at least two weights, Weight is the most familiar axis. But several other axes are 
frequently found in both variable fonts and extended font families. Junicode has Weight and Width axes (Width varying 
from {\jCond 75 Condensed} to {\jExp 125 Expanded}, with 100 Regular in the middle), and the variable font also has 
an Enlarge axis, which can vary the size of many lowercase letters from that of the font's capitals to that of the 
lowercase letters:
\begin{figure}[h!]
  \centering\includegraphics[width=4in]{dns.png}
\end{figure}
Just as the size of these sentence-initial letters varies widely in manuscripts, so it can vary on web pages and in 
print (though few applications for producing printed documents currently support variable fonts). Notice that the letters 
are not simply scaled: the proportions change and the weight remains consistent (a lowercase letter scaled up would look 
too heavy, but a letter scaled via the Enlarged axis will have its original weight at the lower end of the axis and the 
same weight as a capital at the top).

The Enlarge axis runs from 0 to 100. You can choose any number in that range:
to match the effect of ss06\index{ss06} precisely, choose 32.
To ensure that the xheight of all letters matches, choose 47 or less: above that value, the xheight of letters like \textex{e} 
increases at a higher rate than that of letters like \textex{b}.

To use the axis in a web page, declare a CSS class specifying the value for the axis. For example, the second of the examples
in the figure above has the axis set to 75:
\begin{verbatim}
  .SentenceInitial {
    font-variation-settings: "wght" 400, "wdth" 100, "ENLA" 75;
  }
\end{verbatim}
\noindent In the text, enclose the first letter of a sentence in a \verb!<span>! with the class “Sentence\-Initial” (the entity 
is for insular d):
\begin{verbatim}
  <span class="SentenceInitial">&#xA77A;</span>ñs
\end{verbatim}
\noindent The result will be an abbreviation that begins with an “Enlarged Minuscule” insular d, precisely matching the look 
of the second example in the figure above.

These lowercase letters are affected by the Enlarge axis:\footnote{\ Note that all composite characters (e.g. \textex{á}, \textex{ü})
based on these are also affected, so that the actual number of affected characters is much greater than shown here.}
\begin{multicols}{6}
  a\hfill→\hfill\enlax{a}

  \cvd[1]{2}{a}\hfill→\hfill\enlax{\cvd[1]{2}{a}}

  ꜳ\hfill→\hfill\enlax{ꜳ}

  \cvd{55}{ꜳ}\hfill→\hfill\enlax{\cvd{55}{ꜳ}}

  æ\hfill→\hfill\enlax{æ}

  \cvd[3]{57}{æ}\hfill→\hfill\enlax{\cvd[3]{57}{æ}}

  ꜵ\hfill→\hfill\enlax{ꜵ}

  \cvd{59}{ꜵ}\hfill→\hfill\enlax{\cvd{59}{ꜵ}}

  \cvd[1]{59}{ꜵ}\hfill→\hfill\enlax{\cvd[1]{59}{ꜵ}}

  ꜷ\hfill→\hfill\enlax{ꜷ}

  ꜹ\hfill→\hfill\enlax{ꜹ}

  ꜻ\hfill→\hfill\enlax{ꜻ}

  ꜽ\hfill→\hfill\enlax{ꜽ}

  b\hfill→\hfill\enlax{b}

  c\hfill→\hfill\enlax{c}

  d\hfill→\hfill\enlax{d}

  đ\hfill→\hfill\enlax{đ}

  ꝺ\hfill→\hfill\enlax{ꝺ}

  {\addfontfeature{Language=Icelandic}ð}\hfill→\hfill\enlax{{\addfontfeature{Language=Icelandic}ð}}

  {\addfontfeature{Language=English}ð}\hfill→\hfill\enlax{{\addfontfeature{Language=English}ð}}

  e\hfill→\hfill\enlax{e}

  ȩ\hfill→\hfill\enlax{ȩ}

  ę\hfill→\hfill\enlax{ę}

  \cvd{62}{ę}\hfill→\hfill\enlax{\cvd{62}{ę}}

  \cvd[1]{62}{ę}\hfill→\hfill\enlax{\cvd[1]{62}{ę}}

  f\hfill→\hfill\enlax{f}

  \cvd[4]{12}{f}\hfill→\hfill\enlax{\cvd[4]{12}{f}}

  ꝼ\hfill→\hfill\enlax{ꝼ}

  g\hfill→\hfill\enlax{g}

  ꟑ\hfill→\hfill\enlax{ꟑ}

  ᵹ\hfill→\hfill\enlax{ᵹ}

  h\hfill→\hfill\enlax{h}

  \cvd{16}{h}\hfill→\hfill\enlax{\cvd{16}{h}}

  \cvd[3]{16}{h}\hfill→\hfill\enlax{\cvd[3]{16}{h}}

  ħ\hfill→\hfill\enlax{ħ}

  \cvd[4]{16}{h}\hfill→\hfill\enlax{\cvd[4]{16}{h}}

  i\hfill→\hfill\enlax{i}

  ı\hfill→\hfill\enlax{ı}

  j\hfill→\hfill\enlax{j}

  ȷ\hfill→\hfill\enlax{ȷ}

  k\hfill→\hfill\enlax{k}

  l\hfill→\hfill\enlax{l}

  ł\hfill→\hfill\enlax{ł}

  m\hfill→\hfill\enlax{m}

  n\hfill→\hfill\enlax{n}

  \cvd{28}{n}\hfill→\hfill\enlax{\cvd{28}{n}}

  o\hfill→\hfill\enlax{o}

  ɵ\hfill→\hfill\enlax{ɵ}

  ơ\hfill→\hfill\enlax{ơ}

  ƣ\hfill→\hfill\enlax{ƣ}

  ꝋ\hfill→\hfill\enlax{ꝋ}

  ꝏ\hfill→\hfill\enlax{ꝏ}

  ǫ\hfill→\hfill\enlax{ǫ}

  ø\hfill→\hfill\enlax{ø}

  œ\hfill→\hfill\enlax{œ}

  p\hfill→\hfill\enlax{p}

  ꝓ\hfill→\hfill\enlax{ꝓ}

  ꝕ\hfill→\hfill\enlax{ꝕ}

  ꝑ\hfill→\hfill\enlax{ꝑ}

  q\hfill→\hfill\enlax{q}

  ꝙ\hfill→\hfill\enlax{ꝙ}

  r\hfill→\hfill\enlax{r}

  ꞃ\hfill→\hfill\enlax{ꞃ}

  ꝛ\hfill→\hfill\enlax{ꝛ}

  ꝝ\hfill→\hfill\enlax{ꝝ}

  s\hfill→\hfill\enlax{s}

  ꞅ\hfill→\hfill\enlax{ꞅ}

  t\hfill→\hfill\enlax{t}

  ꞇ\hfill→\hfill\enlax{ꞇ}

  u\hfill→\hfill\enlax{u}

  v\hfill→\hfill\enlax{v}

  w\hfill→\hfill\enlax{w}

  ƿ\hfill→\hfill\enlax{ƿ}

  x\hfill→\hfill\enlax{x}

  y\hfill→\hfill\enlax{y}

  z\hfill→\hfill\enlax{z}

  {\addfontfeature{Language=Icelandic}þ}\hfill→\hfill\enlax{{\addfontfeature{Language=Icelandic}þ}}

  {\addfontfeature{Language=English}þ}\hfill→\hfill\enlax{{\addfontfeature{Language=English}þ}}

  {\addfontfeature{Language=Icelandic}ꝥ}\hfill→\hfill\enlax{{\addfontfeature{Language=Icelandic}ꝥ}}

  {\addfontfeature{Language=English}ꝥ}\hfill→\hfill\enlax{{\addfontfeature{Language=English}ꝥ}}

  {\addfontfeature{Language=Icelandic,CharacterVariant=66}ꝥ}\hfill→\hfill\enlax{{\addfontfeature{Language=Icelandic,CharacterVariant=66}ꝥ}}

  {\addfontfeature{Language=English,CharacterVariant=66}ꝥ}\hfill→\hfill\enlax{{\addfontfeature{Language=English,CharacterVariant=66}ꝥ}}

  ꝧ\hfill→\hfill\enlax{ꝧ}
\end{multicols}